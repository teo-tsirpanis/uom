\documentclass{article}

\usepackage{amsmath}
\usepackage{amssymb}
\usepackage{booktabs}

\usepackage{hyperref}
\usepackage{fontspec}
\usepackage{longtable}
\usepackage[greek,english]{babel}

\setmainfont{Ubuntu}
\setmonofont{Ubuntu Mono}

\title{Κρυπτογραφία--Άσκηση 8: Ανταλλαγή κλειδιών Diffie-Hellman}
\author{Τσιρπάνης Θεόδωρος\\ \texttt{dai19090}}
\date{Δεκέμβριος 2019}

\begin{document}

\selectlanguage{greek}

\maketitle

\section*{Ερώτημα 1}

Το πρώτο ερώτημα μας ζητάει να υπολογίσουμε την τάξη όλων των στοιχείων των πολλαπλασιαστικών ομάδων $\mathbb{Z}_5^*$, $\mathbb{Z}_7^*$, $\mathbb{Z}_{13}^*$. Θα δουλέψουμε ως εξής. Για κάθε ομάδα $\mathbb{Z}_n^*$ θα φτιάξουμε έναν πίνακα όπου κάθε κελί του θα έχει την πράξη $y^x \pmod n$, όπου $(x, y) \in {\mathbb{Z}_n^*}^2$ οι συντεταγμένες του κελιού. Αντί για $\mathcal{O}(n^2)$ εκθετοποιήσεις, θα κάνουμε τον ίδιο αριθμό \emph{πολλαπλασιασμών}, αφού γνωρίζουμε ότι $y^x = x \cdot y^{x - 1}$.

\begin{longtable}[c]{l|l|l|l|l|l}
    $i$ & 0 & 1 & 2 & 3 & 4 \\ \hline
    ${1}^i \pmod 5$ & 1 & 1 & 1 & 1 & 1 \\ \hline
    ${2}^i \pmod 5$ & 1 & 2 & 4 & 3 & 1 \\ \hline
    ${3}^i \pmod 5$ & 1 & 3 & 4 & 2 & 1 \\ \hline
    ${4}^i \pmod 5$ & 1 & 4 & 1 & 4 & 1 \\
\end{longtable}

\begin{longtable}[c]{l|l|l|l|l|l|l|l}
    $i$ & 0 & 1 & 2 & 3 & 4 & 5 & 6 \\ \hline
    ${1}^i \pmod 7$ & 1 & 1 & 1 & 1 & 1 & 1 \\ \hline
    ${2}^i \pmod 7$ & 2 & 4 & 1 & 2 & 4 & 1 \\ \hline
    ${3}^i \pmod 7$ & 3 & 2 & 6 & 4 & 5 & 1 \\ \hline
    ${4}^i \pmod 7$ & 4 & 2 & 1 & 4 & 2 & 1 \\ \hline
    ${5}^i \pmod 7$ & 5 & 4 & 6 & 2 & 3 & 1 \\ \hline
    ${6}^i \pmod 7$ & 6 & 1 & 6 & 1 & 6 & 1 \\
\end{longtable}

\begin{longtable}[c]{l|l|l|l|l|l|l|l|l|l|l|l|l|l}
    $i$ & 0 & 1 & 2 & 3 & 4 & 5 & 6 & 7 & 8 & 9 & 10 & 11 & 12 \\ \hline
    ${ 1}^i \pmod {13}$ & 1 & 1 & 1 & 1 & 1 & 1 & 1 & 1 & 1 & 1 & 1 & 1 & 1 \\ \hline
    ${ 2}^i \pmod {13}$ & 1 & 2 & 4 & 8 & 3 & 6 & 12 & 11 & 9 & 5 & 10 & 7 & 1 \\ \hline
    ${ 3}^i \pmod {13}$ & 1 & 3 & 9 & 1 & 3 & 9 & 1 & 3 & 9 & 1 & 3 & 9 & 1 \\ \hline
    ${ 4}^i \pmod {13}$ & 1 & 4 & 3 & 12 & 9 & 10 & 1 & 4 & 3 & 12 & 9 & 10 & 1 \\ \hline
    ${ 5}^i \pmod {13}$ & 1 & 5 & 12 & 8 & 1 & 5 & 12 & 8 & 1 & 5 & 12 & 8 & 1 \\ \hline
    ${ 6}^i \pmod {13}$ & 1 & 6 & 10 & 8 & 9 & 2 & 12 & 7 & 3 & 5 & 4 & 11 & 1 \\ \hline
    ${ 7}^i \pmod {13}$ & 1 & 7 & 10 & 5 & 9 & 11 & 12 & 6 & 3 & 8 & 4 & 2 & 1 \\ \hline
    ${ 8}^i \pmod {13}$ & 1 & 8 & 12 & 5 & 1 & 8 & 12 & 5 & 1 & 8 & 12 & 5 & 1 \\ \hline
    ${ 9}^i \pmod {13}$ & 1 & 9 & 3 & 1 & 9 & 3 & 1 & 9 & 3 & 1 & 9 & 3 & 1 \\ \hline
    ${10}^i \pmod {13}$ & 1 & 10 & 9 & 12 & 3 & 4 & 1 & 10 & 9 & 12 & 3 & 4 & 1 \\ \hline
    ${11}^i \pmod {13}$ & 1 & 11 & 4 & 5 & 3 & 7 & 12 & 2 & 9 & 8 & 10 & 6 & 1 \\ \hline
    ${12}^i \pmod {13}$ & 1 & 12 & 1 & 12 & 1 & 12 & 1 & 12 & 1 & 12 & 1 & 12 & 1 \\
\end{longtable}

\section*{Ερώτημα 2}

Στο δεύτερο ερώτημα θα δημιουργήσουμε μερικά κλειδιά συνεδρίας με το πρωτόκολλο Diffie-Hellman. Για όλες τις περιπτώσεις, ισχύει ότι η ομάδα στην οποία κάνουμε πράξεις είναι η $\mathbb{Z}_{467}^*$, και ο γεννήτοράς της είναι ο αριθμός $g = 2$. Οι μυστικές παράμετροι της Alice και του Bob συμβολίζονται με $a$ και $b$ αντίστοιχα.

\begin{itemize}
    \item Έχουμε $(a, b) = (3, 5)$. Η Alice δημιουργεί το $s_a = 2^3 \equiv 8 \pmod{467}$ και το στέλνει στον Bob. Αυτός υπολογίζει το $s_b = 2^5 \equiv 32 \pmod{467}$ και το στέλνει στην Alice.

    Οπλισμένη με το $s_b$, η Alice μπορεί να βρει το $s = 32^3 \equiv 78 \pmod{467}$, και ο Bob θα βρει με τη σειρά του το $s = 8^5 \equiv 78 \pmod{467}$. Κάπως έτσι η Alice και ο Bob, καθιέρωσαν ένα μυστικό κοινό μόνο γι' αυτούς, για να επικοινωνήσουν με ασφάλεια.

    \item Έχουμε $(a, b) = (400, 134)$. Άρα, $s_a = 2^{400} \equiv 137 \pmod{467}$, $s_b = 2^{134} \equiv 84 \pmod{467}$, και το μυστικό κλειδί είναι $s_b^a = 84^{400} \equiv 90 \equiv 137^{134} = s_a^b \pmod{467}$.

    \item Έχουμε $(a, b) = (228, 57)$. Άρα, $s_a = 2^{228} \equiv 394 \pmod{467}$, $s_b = 2^{57} \equiv 313 \pmod{467}$, και το μυστικό κλειδί είναι $s_b^a = 313^{228} \equiv 206 \equiv 394^{57} = s_a^b \pmod{467}$.
\end{itemize}

\section*{Ερώτημα 3}

Στο τρίτο ερώτημα η Alice και ο Bob θα ανταλλάξουν ένα μυστικό πάλι με το πρωτόκολλο Diffie-Hellman, αλλά με μια σημαντική παραλλαγή. Αντί για μια ομάδα με πρώτη τάξη, θα κάνομυε τις πράξεις μας στο σώμα $GF(2^5)$, με ανάγωγο πολυώνυμο το $P(x) = x^5 + x^2 + 1$ και γεννήτορα το $g = x^2$. Η Alice έχει τον μυστικό εκθέτη $a = 3$, και ο Bob το $b = 12$.

Η Alice ξεκινάει υπολογίζοντας το $s_a$:

\begin{align*}
    s_a
    &= g^a \\
    &= (x^2)^3 \\
    &= x^6 \\
    &= x(x^5) \\
    &= x(x^2 + 1) \\
    &= x^3 + x \\
\end{align*}

Το ίδιο θα κάνει και ο Bob με το $s_b$:

\begin{align*}
    s_b
    &= g^b \\
    &= (x^2)^12 \\
    &= x^24 \\
    &= (x^6)^4 \\
    &= (x^3 + x)^4 \\
    &= (x^6 + x^2)^2 \\
    &= (x^3 + x^2 + x)^2 \\
    &= x^6 + x^4 + x^2 \\
    &= x^4 + x^3 + x^2 + x \\
\end{align*}

Ο Bob μπορεί να μη το ήξερε, αλλά εμείς ξέραμε ήδη πόσο κάνει $x^6$ και αποφύγαμε μερικές πράξεις. Η Alice με τον Bob ανταλλάσσουν τις τιμές που μόλις υπολογίσαν, και βρίσκουν το μυστικό κλειδί $s$ ως εξής:

\begin{align*}
    s_a^b
    &= (x^3 + x)^12 \\
    &= (((x^3 + x)^2)^2)^3 \\
    &= ((x^6 + x^2)^2)^3 \\
    &= ((x^3 + x^2 + x)^2)^3 \\
    &= (x^6 + x^4 + x^2)^3 \\
    &= (x^4 + x^3 + x^2 + x)^3 \\
    &= (x^4 + x^3 + x^2 + x)(x^4 + x^3 + x^2 + x)^2 \\
    &= (x^4 + x^3 + x^2 + x)(x^8 + x^6 + x^4 + x^2) \\
    &= (x^4 + x^3 + x^2 + x)(x^6 \cdot x^2 + x^6 + x^4 + x^2) \\
    &= (x^4 + x^3 + x^2 + x)((x^3 + x) \cdot x^2 + x^6 + x^4 + x^2) \\
    &= (x^4 + x^3 + x^2 + x)(x^5 + x^3 + x^6 + x^4 + x^2) \\
    &= (x^4 + x^3 + x^2 + x)(x^2 + 1 + x^3 + x^3 + x + x^4 + x^2) \\
    &= (x^4 + x^3 + x^2 + x)(x^4 + x + 1) \\
    &= x^8 + x^7 + x^6 + x^5 + x^5 + x^4 + x^3 + x^2 + x^4 + x^3 + x^2 + x \\
    &= x^8 + x^7 + x^6 + x \\
    &= x^5 + x^3 + x^4 + x^2 + x^3 + x + x \\
    &= x^5 + x^4 + x^2 \\
    &= x^2 + 1 + x^4 + x^2 \\
    &= x^4 + 1 \\
    &= (x^4 + x^3 + x^2 + x)(x^4 + x + 1) \\
    &= (x^4 + x^3 + x^2 + x)(x^2 + 1 + x + x^4 + x^2) \\
    &= (x^4 + x^3 + x^2 + x)(x^5 + x^3 + x^3 + x + x^4 + x^2) \\
    &= (x^4 + x^3 + x^2 + x)(x^8 + x^6 + x^4 + x^2) \\
    &= (x^4 + x^3 + x^2 + x)(x^4 + x^3 + x^2 + x)^2 \\
    &= (x^4 + x^3 + x^2 + x)^3 \\
    &= s_b^a
\end{align*}

Επομένως, το μυστικό $s$ που έχουν οι Alice και Bob είναι το πολυώνυμο $x^4 + 1$. Ορισμένες από τις πράξεις που καναμε παραπάνω απαλείφθηκαν, επειδή πραγματοποιήθηκαν πολλές φορές κατά τη διάρκεια των υπολογισμών.

\section*{Ερώτημα 4}

Το τέταρτο ερώτημα αφορά το κρυπτοσύστημα ElGamal στο οποίο θα κρυπτογραφήσουμε ορισμένα μηνύματα. Για όλες τις περιπτώσεις, ισχύει ότι η ομάδα στην οποία κάνουμε πράξεις είναι η $\mathbb{Z}_{467}^*$, και ο γεννήτοράς της είναι ο αριθμός $g = 2$.

\begin{itemize}
    \item Το ιδιωτικό κλειδί είναι το $d = 105$. Αυτό σημαίνει ότι $B = 2^{105} \equiv 444 \pmod{467}$. Με $i = 213$, έχουμε $k_e = g^i = 2^{213} \equiv 29 \pmod{467}$ και $k_s = B^i = 444^{213} \equiv 292 \pmod{467}$. Άρα, κρυπτογραφώντας το μήνυμα $m = 33$, το κρυπτοκείμενο είναι το $c = k_sm = 292 \cdot 33 \equiv 296 \pmod{467}$ μαζί με το $k_e$.

    Ας προσπαθήσουμε να το αποκρυπτογραφήσουμε, γνωρίζοντας το ιδιωτικό κλειδί. Έχουμε $k_s = k_e^d = 29^{105} \equiv 292 \pmod{467}$. Γνωρίζοντάς το, μπορούμε να βρούμε το μήνυμα $m = k_s^{-1}c = 292^{-1} \cdot 296 = 8 \cdot 296 \equiv 33$, το οποίο ήταν αυτό που περιμέναμε.

    \item Ας ξανακάνουμε το παραπάνω παράδειγμα, αλλά αυτήν τη φορά με $i = 123$. Έχουμε $k_e = 2^{123} \equiv 125 \pmod{467}$ και $k_s = 444^{123} \equiv 278 \pmod{467}$. Άρα, $c = 278 \cdot 33 \equiv 301 \pmod{467}$.

    Συνεπώς, $k_s = 125^{105} \equiv 278 \pmod{467}$, και $m = 278^{-1} \cdot 301 = 42 \cdot 301 \equiv 33 \pmod{467}$.

    \item
    \begin{gather*}
        d = 300 \\
        m = 248 \\
        i = 45 \\
        B = 2^{300} \equiv 317 \pmod{467} \\
        \midrule
        k_s = 317^{45} \equiv 12 \pmod{467} \\
        k_e = 2^{45} \equiv 80 \pmod{467} \\
        c = 12 \cdot 248 \equiv 174 \pmod{467} \\
        \midrule
        k_s = 80^{300} \equiv 12 \pmod{467} \\
        k_s^{-1} \equiv 39 \pmod{467} \\
        m = 39 \cdot 174 \equiv 248 \pmod{467}
    \end{gather*}

    \item
    \begin{gather*}
        d = 300 \\
        m = 248 \\
        i = 47 \\
        B = 2^{300} \equiv 317 \pmod{467} \\
        \midrule
        k_s = 317^{47} \equiv 74 \pmod{467} \\
        k_e = 2^{47} \equiv 320 \pmod{467} \\
        c = 74 \cdot 248 \equiv 139 \pmod{467} \\
        \midrule
        k_s = 320^{300} \equiv 74 \pmod{467} \\
        k_s^{-1} \equiv 284 \pmod{467} \\
        m = 284 \cdot 139 \equiv 248 \pmod{467}
    \end{gather*}
\end{itemize}

\section*{Ερώτημα 5}

Στο τελευταίο ερώτημα θα παίζουμε το ρόλο της Eve. Βλέπουμε ότι ο Bob στέλνει στην Alice μηνύματα με το κρυπτοσύστημα ElGamal πάνω στο $\mathbb{Z}_{31}^*$, και θέλουμε να τα βρούμε. Όμως κάνει ένα μοιραίο λάθος. Σε κάθε κρυπτογράφηση, χρησιμοποιεί την ίδια τιμή $i$. Γνωρίζοντάς το (χωρίς να ξέρουμε απαραίτητα \emph{ποιο} είναι το $i$), η κρυπτανάλυση είναι εκπληκτικά απλή, και μπορεί να πραγματοποιηθεί με μια επίθεση γνωστού απλού κειμένου.

Ευτυχώς για 'μας (και όχι για τον Bob), γνωρίζουμε ότι το πρώτο του μήνυμα κάθε φορά είναι ο αριθμός μητρώου του, το $m = 21$, που κρυπτογραφήθηκε στο $c = 17$.

Πάμε στην επίθεση τώρα. Εφόσον το $i$ είναι σταθερό, θα είναι και το $k_s$. Οπότε έχουμε $c = k_s \cdot m \Leftrightarrow k_s = m^{-1}c = 21^{-1} \cdot 17 = 3 \cdot 17 \equiv 20 \pmod{31}$.

Και τώρα ας προσπαθήσουμε να αποκρυπτογραφήσουμε το επόμενο μήνυμα που υποκλέψαμε, το $c = 25$. Έχουμε $m = k_s^{-1}c = 20^{-1} \cdot 25 = 14 \cdot 25 \equiv 9 \pmod{31}$. Επομένως, το δεύτερο μήνυμα που έστειλε ο Bob στην Alice είναι το $m = 9$.

Φάνηκε έτσι ο λόγος που είναι απολύτως αναγκαία η επιλογή διαφορετικού $i$ σε κάθε κρυπτογράφηση. Κάνοντας μια ιστορική αναδρομή, η κονσόλα βιντεοπαιχνιδιών PlayStation 3 χρησιμοποιούσε έναν αλγόριθμο ψηφιακών υπογραφών με ελλειπτικές καμπύλες, τον ECDSA, ο οποίος επίσης απαιτεί έναν διαφορετικό τυχαίο αριθμό για κάθε υπογραφή. Τα συστήματά ασφαλείας του λοιπόν, μπόρεσαν να παρακαμφθούν, όταν ανακαλύφθηκε ότι ο αριθμός αυτός ήταν ο ίδιος για όλες τις υπογραφές, \href{https://www.exophase.com/20540/hackers-describe-ps3-security-as-epic-fail-gain-unrestricted-access/}{επιτρέποντας σε ερασιτέχνες προγραμματιστές να αναπτύσσουν και να εκτελούν μη εξουσιοδοτημένο από τη Sony κώδικα στις κονσόλες τους (homebrew)}.

\end{document}
