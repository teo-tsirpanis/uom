\documentclass{article}
\usepackage{fontspec}
\usepackage{listings}
\lstloadlanguages{Python}
\lstset{
    language=Python,
    basicstyle=\small\ttfamily,
    breakatwhitespace=false,
    breaklines=true,
    captionpos=b,
    numbers=left,
    numbersep=5pt,
    keepspaces=true,
    showstringspaces=false,
    tabsize=2}

\setmainfont{Ubuntu}
\setmonofont{Ubuntu Mono}

\title{Κρυπτογραφία ~ Άσκηση 2: Επίθεση μόνο Κρυπτοκειμένου}
\author{Τσιρπάνης Θεόδωρος\\ \texttt{dai19090}}
\date{Οκτώβριος 2019}

\begin{document}

\maketitle

\section{Ερώτημα 1}

Για να χρησιμοποιήσουμε τους δακτυλίους του SageMath στον ομοπαραλληλικό αλγόριθμο, αρκεί:

\begin{itemize}
\item να μετατρέψουμε τους αριθμούς στην συνάρτηση \lstinline|str2lst| ώστε να ανήκουν στον δακτύλιο $Z_n$.

\item να γυρίσουμε τους άκεραίους στην συνάρτηση \lstinline|lst2str| πίσω στον $Z_n$.

\item να αφαιρέσουμε τον τελεστή modulo στις συναρτήσεις κρυπτογράφισης και αποκρυπτογράφισης.
\end{itemize}


Με τις παραπάνω αλλαγές, ο κώδικας θα γίνει ως εξής:

\begin{lstlisting}[caption=Ο κώδικας στο SageMath]
n = 26
Zn = IntegerModRing(n)

a=11
b=4

def str2lst(s):
    return [Zn(ord(x)-ord(``A'')) for x in s]

def lst2str(lst):
    return ''.join([chr(int(x)+ord(``A'')) for x in lst])

def affine_enc(m,k1,k2):
    plaintextList = str2lst(m)
    ciphertextList = [k1*x + k2 for x in plaintextList]
    ciphertext = lst2str(ciphertextList)
    return ciphertext

def affine_dec(c,k1,k2):
    k1_inverse=inverse_mod(k1,26)
    ciphertextList = str2lst(c)
    plaintextList = [k1_inverse * (x-k2) for x in ciphertextList]
    plaintext = lst2str(plaintextList)
    return plaintext
\end{lstlisting}

Ένα άλλο πλεονέκτημα στην μετατροπή αυτή είναι ότι οι συναρτήσεις \lstinline|affine_enc| και \lstinline|affine_dec| λειτουργούν ανεξάρτητα από την τάξη του δακτυλίου. Η αρμοδιότητα αυτή βρίσκεται μόνο στην \lstinline|str2lst|.



\end{document}
