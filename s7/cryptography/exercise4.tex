\documentclass{article}

\usepackage{amsmath}
\usepackage{amssymb}

\usepackage{fontspec}
\usepackage{longtable}

\usepackage{listings}
\lstloadlanguages{Python}
\lstset{
    language=Python,
    basicstyle=\small\ttfamily,
    breakatwhitespace=false,
    breaklines=true,
    captionpos=b,
    numbers=left,
    numbersep=5pt,
    keepspaces=true,
    showstringspaces=false,
    tabsize=2}

\setmainfont{Ubuntu}
\setmonofont{Ubuntu Mono}

\title{Κρυπτογραφία--Άσκηση 4: Το Κρυπτοσύστημα Pohlig - Hellman}
\author{Τσιρπάνης Θεόδωρος\\ \texttt{dai19090}}
\date{Νοέμβριος 2019}

\begin{document}

\maketitle

\section*{Ερώτημα 1}

Το ερώτημα μας ζητάει να κρυπτογραφήσουμε και να αποκρυπτογραφήσουμε με το κρυπτοσύστημα Pohlig-Hellman το μήνυμα \texttt{SWEETDREAMS} με κλειδί το $e = 7$ και παράμετρο $p = 2621$. Για να ελαχιστοποιήσουμε τις πράξεις που θα κάνουμε, θα χωρίσουμε το κρυπτοκείμενο σε τμήματα μεγέθους δύο γραμμάτων. Αλλά πριν το κάνουμε, ας βρούμε το κλειδί αποκρυπτογράφησης. Η εύρεσή του είναι εκπληκτικά απλή. Το μόνο που έχουμε να κάνουμε είναι να εκτελέσουμε στο SageMath την εντολή \lstinline|inverse_mod(7, 2621 - 1)|, που θα μας πει ότι το κλειδί αποκρυπτογράφησης είναι $d = 1123$. Έχοντας βρει το $d$, είμαστε έτοιμοι να συνεχίσουμε:

\begin{longtable}[c]{|l|l|l|l|l|l|l|}
\hline
& SW & EE & TD & RE & AM & S \\
\hline
& $1822$ & $404$ & $1903$ & $1704$ & $12$ & $1800$ \\
\hline
$c \equiv m^e \pmod p$ & $394$ & $1679$ & $1804$ & $755$ & $117$ & $319$ \\
\hline
$m \equiv c^d \pmod p$ & $1822$ & $404$ & $1903$ & $1704$ & $12$ & $1800$ \\
\hline
\end{longtable}

Αν δούμε τους αριθμούς στην δεύτερη και την τέταρτη σειρά, παρατηρούμε ότι είναι ίσοι ανά στήλη, που σημαίνει ότι το κρυπτοσύστημά μας δουλεύει, αφού είναι αντιστρέψιμο, όπως θα περιμέναμε από ένα κρυπτοσύστημα να είναι.

\section*{Ερώτημα 2}

Συνεχίζουμε με το ίδιο κρυπτοσύστημα, αυτήν τη φορά με άγνωστα κλειδιά $e, d$, και $p = 29$. Γνωρίζουμε επίσης ότι $d(24) = 20$. Αυτά μας αρκούν για να βρούμε το $d$ που είναι $\log_{24}(20) \pmod p$. Μπορούμε ευτυχώς να υπολογίσουμε τον διακριτό λογάριθμο με τον ακόλουθο κώδικα σε Python:

\begin{lstlisting}
def discrete_log(m, c, p):
i = 1
x = m
while x != c:
    x = (x * m) % p
    i += 1
return i, inverse_mod(i, p - 1)
print discrete_log(20, 24, 29)
\end{lstlisting}

Ο κώδικας μας βγάζει ότι $(e, d) = (5, 17)$, κάτι που ισχύει, καθώς $20^5 \equiv 24 \pmod{29}$ και $24^{17} \equiv 20 \pmod{29}$.

Γνωρίζοντας το κλειδί, πάμε να αποκρυπτογραφήσουμε ολόκληρο το μήνυμα:

\begin{longtable}[c]{|l|l|l|l|l|l|l|l|l|l|}
\hline
& $04$ & $19$ & $19$ & $11$ & $04$ & $24$ & $09$ & $15$ & $15$ \\
\hline
$m \equiv c^{17} \pmod{29}$ & $06$ & $14$ & $14$ & $03$ & $06$ & $20$ & $04$ & $18$ & $18$ \\
\hline
& G & O & O & D & G & U & E & S & S \\
\hline
\end{longtable}

Όπως φάνηκε, το μήνυμα ήταν μια καλή μαντεψιά.

\section*{Ερώτημα 3}

Ας δούμε ένα μεγαλύτερο παράδειγμα. Έχουμε ένα κρυπτοσύστημα Pohlig-Hellman, με $p = 2591$ και γνωρίζοντας ότι $d(0902) = 1314$. Το κλειδί είναι πάλι άγνωστο, αλλά μεγαλύτερο από πριν, αν και όχι τόσο μεγάλο για να σταθεί εμπόδιο στην εξαντλητική αναζήτηση που περιγράψαμε στον παραπάνω κώδικα. Μια κλήση της συνάρτησης \lstinline|discrete_log(1314, 902, 2591)| θα επιστρέψει πολύ γρήγορα το ζεύγος $(e, d) = (13, 797)$. 

\begin{longtable}[c]{|l|l|l|l|l|l|l|l|}
\hline
& $1213$ & $0902$ & $0539$ & $1208$ & $1234$ & $1103$ & $1374$ \\
\hline
$m \equiv c^{797} \pmod{2591}$ & $314$ & $1314$ & $1917$ & $400$ & $319$ & $708$ & $1823$ \\
\hline
& DO & NO & TR & EA & DT & HI & SX \\
\hline
\end{longtable}

Όπως φάνηκε, δεν έπρεπε να διαβάσουμε το μήνυμα.

\section*{Ερώτημα 4}

Ένα πρόβλημα με το Κρυπτοσύστημα Pohlig-Hellman είναι ότι αν το μήνυμα $m$ δεν είναι γεννήτορας του $\mathbb{Z}_p$, ο διακριτός λογάριθμος έχει πολλές λύσεις. Αυτό μας επιτρέπει να αποκλείσουμε κάποιους αριθμούς ως πιθανά κλειδιά. Για παράδειγμα, αν το μήνυμα $m$ έχει τάξη $2$ στο $\mathbb{Z}_p$, το $c \equiv m^e \pmod p$ μπορεί να πάρει μόνο δύο διαφορετικές τιμές. Γνωρίζοντας το $c$ και το $m$, συμπεράναμε ότι από όλα τα πιθανά κλειδιά, τα μισά σίγουρα δεν είναι το $e$. Έχοντας ολοένα και περισσότερα ζεύγη $(m, c)$ με το $m$ να μην είναι γεννήτορας, το πλήθος των πιθανών $e$ ολοένα και μικραίνει. Για μικρά $p$, ο διακριτός λογάριθμος μπορεί να βρεθεί με αλγορίθμους εξαντλητικής αναζήτησης, όπως αυτόν που περιγράφεται στο απόσπασμα κώδικα του Ερωτήματος 2.

\end{document}
