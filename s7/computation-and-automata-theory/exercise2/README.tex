\documentclass{article}

\usepackage{amsmath}
\usepackage{amssymb}
\usepackage{booktabs}

\usepackage{hyperref}
\usepackage{fontspec}
\usepackage{longtable}
\usepackage[greek,english]{babel}

\setmainfont{Ubuntu}
\setmonofont{Ubuntu Mono}

\title{Θεωρία Υπολογισμών και Αυτομάτων\\ Άσκηση 2}
\author{Τσιρπάνης Θεόδωρος\\ \texttt{dai19090}}
\date{Ιανουάριος 2020}

\begin{document}

\selectlanguage{greek}

\maketitle

\section{Προσθήκη στο δυαδικό σύστημα}

Η άσκηση ζητάει την κατασκευή μιας μηχανής Turing η οποία θα προσθέτει δύο αριθμούς στο δυαδικό σύστημα. Πριν την περιγράψουμε, ας θυμηθούμε τους κανόνες της πρόσθεσης. Η συνάρτηση της πρόσθεσης έχει είσοδο δύο μπιτ $b_1$ και $b_2$, και το μπιτ $c$ ως το κρατούμενο από την προηγούμενη πρόσθεση, επιστρέφει ένα αποτέλεσμα $x$, και ένα νέο κρατούμενο $c'$. Το αποτέλεσμα είναι $1$, αν και μόνο αν το πλήθος των μπιτ εισόδου που είναι ίσα με το $1$, είναι περιττός αριθμός (δηλαδή μόνο ένα ή και τα τρία). Το νέο κρατούμενο είναι $1$, αν και μόνο αν το πλήθος των μπιτ εισόδου που είναι ίσα με το $1$ είναι μεγαλύτερο ή ίσο του δύο (δηλαδή δύο ή και τα τρία). Σε αντίθετη περίπτωση, το αντίστοιχο μπιτ εξόδου είναι $0$. Ας κάνουμε και έναν πίνακα αληθείας της συνάρτησής μας.

\begin{longtable}[c]{|l|l|l||l|l|}
    \hline
    $b_1$ & $b_2$ & $c$ & $x$ & $c'$ \\
    \hline
    $0$ & $0$ & $0$ & $0$ & $0$ \\ \hline
    $0$ & $0$ & $1$ & $1$ & $0$ \\ \hline
    $0$ & $1$ & $0$ & $1$ & $0$ \\ \hline
    $0$ & $1$ & $1$ & $0$ & $1$ \\ \hline
    $1$ & $0$ & $0$ & $1$ & $0$ \\ \hline
    $1$ & $0$ & $1$ & $0$ & $1$ \\ \hline
    $1$ & $1$ & $0$ & $0$ & $1$ \\ \hline
    $1$ & $1$ & $1$ & $1$ & $1$ \\ \hline
\end{longtable}

Η πρόσθεση λέξεων με πολλά μπιτ γίνεται προσθέτοντας ένα-ένα τα μπιτ της λέξης σύμφωνα την συνάρτηση, ενημερώνοντας το κρατούμενο σε κάθε βήμα, ξεκινώντας από το λιγότερο σημαντικό μπιτ των δύο λέξεων και με το κρατούμενο να είναι ίσο με $0$. Στο τέλος, τοποθετείται στην αρχή του αποτελέσματος, το μπιτ του κρατουμένου.

\section{Η μηχανή Turing}

Η μηχανή που θα φτιάξουμε θα έχει τρεις ταινίες. Στις δύο θα υπάρχουν οι προσθετέοι, και η τρίτη που θα είναι στην αρχή κενή, θα έχει στο τέλος το αποτέλεσμα. Θα έχει τρεις καταστάσεις, μαζί με την κατάσταση τερματισμού. Οι άλλες δύο θα συμβολίζουν την παρουσία ή την απουσία του κρατουμένου, με την δεύτερη εκ των οποίων να είναι η αρχική.

Στις ταινίες, το λιγότερο σημαντικό μπιτ θα βρίσκεται στα δεξιά. Ο δείκτης θα ξεκινάει στο λιγότερο σημαντικό μπιτ κάθε ταινίας και σε κάθε βήμα, οι ταινίες θα μετακινούνται αριστερά, μέχρι το τέλος όπου ο δείκτης θα βρίσκεται στο πιο σημαντικό μπιτ κάθε ταινίας.

Σε κάθε βήμα, το σύμβολο που θα γράφεται στην ταινία εξόδου θα υπολογίζεται βάσει του παραπάνω πίνακα. Επίσης θα γίνεται μετάβαση στην επόμενη κατάσταση, αντικατοπτρίζοντας το νέο κρατούμενο. Το περιεχόμενο των ταινιών εισόδου δεν θα αλλάζει.

Στην περίπτωση που οι λέξεις εισόδου δεν έχουν το ίδιο μήκος, δεν υπάρχει πρόβλημα. Αν η μηχανή μας συναντήσει το κενό σύμβολο σε \emph{μόνο μια} ταινία, δεν θα την μετακινήσει αριστερά, και κατά τ' άλλα θα την αντιμετωπίσει σαν να είχε συναντήσει σε αυτήν το σύμβολο $0$.

Αν η μηχανή μας συναντήσει το κενό σύμβολο \emph{και στις δυο} ταινίες, θα τις μετακινήσει δεξιά (για να φέρει τον δείκτη στο πιο σημαντικό μπιτ), θα γράψει $1$ ή $0$ στην ταινία εξόδου ανάλογα με το αν βρίσκεται στην κατάσταση κρατουμένου ή όχι, και θα μεταβεί στην κατάσταση τερματισμού τελειώνοντας την εκτέλεση.

\section{Η υλοποίηση στο JFLAP}

Η μηχανή μας υλοποιήθηκε στο JFLAP, και επισυνάπτεται μαζί με την εργασία.

Ένα πρόβλημα που μας παρουσιάζεται είναι ότι το JFLAP τοποθετεί τον δείκτη κάθε ταινίας στην \emph{αρχή}, αντί για το τέλος που θέλουμε εμείς. Η λύση του όμως είναι εκπληκτικά απλή. Η αρχική κατάσταση στο JFLAP θα είναι μια καινούργια κατάσταση, η οποία θα μετακινεί χωρίς να τροποποιεί τις ταινίες εισόδου στα δεξιά όσο συναντάει σύμβολα σε κάποια από τις δύο. Έχοντας τελειώσει και με τις δύο, θα μεταβούμε στην κατάσταση χωρίς κρατούμενο, όντας πια έτοιμοι για την πρόσθεση.

\end{document}
