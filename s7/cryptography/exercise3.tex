\documentclass{article}

\usepackage{amsmath}
\usepackage{amssymb}

\usepackage{fontspec}
\usepackage{longtable}
\usepackage{hyperref}

\setmainfont{Ubuntu}
\setmonofont{Ubuntu Mono}

\title{Κρυπτογραφία--Άσκηση 3: Μαθηματικό Υπόβαθρο}
\author{Τσιρπάνης Θεόδωρος\\ \texttt{dai19090}}
\date{Νοέμβριος 2019}

\begin{document}

\maketitle

\section*{Ερώτημα 1}

Το πρώτο ερώτημα μας καλεί να υπολογίσουμε μερικές πράξεις στο $\mathbb{Z}_{13}$:

\begin{enumerate}
    \item $15 \cdot 29 \equiv 2 \cdot 3 \equiv 6 \pmod{13}$
    \item $2 \cdot 29 \equiv 2 \cdot 3 \equiv 6 \pmod{13}$
    \item $2 \cdot 3 \equiv 6 \pmod{13}$
    \item $-11 \cdot 3 \equiv 2 \cdot 3  \equiv 6 \pmod{13}$
\end{enumerate}

Παρατηρούμε ότι εφαρμόζοντας τον τελεστή modulo στα επιμέρους μέλη όλων των πολλαπλασιασμών, καταλήγουμε πάντα στην πράξη $2 \cdot 3 \pmod{13}$, της οποίας το αποτέλεσμα είναι ο αριθμός $6$.

\section*{Ερώτημα 2}

Στο δεύτερο ερώτημα, θα πρέπει να υπολογίσουμε τους αντιστρόφους του $5$ στα $\mathbb{Z}_{11}, \mathbb{Z}_{12}, \mathbb{Z}_{13}$. Επειδή το $5$ είναι πρώτος αριθμός, θα υπάρχουν και οι τρεις αντίστροφοι. Για την εύρεσή τους χρησιμοποιήθηκε διαδικτυακή εφαρμογή\cite{inversemod}.

\begin{enumerate}
    \item Ισχύει $5^{-1} \equiv 9 \pmod{11}$, μιας και $5 \cdot 9 = 45 = 4 \cdot 11 + 1$.
    \item Ισχύει $5^{-1} \equiv 5 \pmod{12}$, μιας και $5 \cdot 5 = 25 = 2 \cdot 12 + 1$.
    \item Ισχύει $5^{-1} \equiv 8 \pmod{13}$, μιας και $5 \cdot 8 = 40 = 3 \cdot 13 + 1$.
\end{enumerate}

\section*{Ερώτημα 3}

Το τρίτο ερώτημα μας ζητάει υπολογίσουμε την συνάρτηση $\phi$ του Euler για τέσσερεις αριθμούς. Ας ξαναθυμηθούμε ότι η $\phi$ αντιστοιχίζει σε έναν θετικό ακέραιο το πλήθος των θετικών ακεραίων που είναι σχετικά πρώτοι με αυτόν. Γνωρίζοντας αυτό, έχουμε:

\begin{enumerate}
    \item Ισχύει $\phi(4) = 2$. Οι σχετικά πρώτοι αριθμοί είναι οι $\{1, 3\}$.
    \item Ισχύει $\phi(5) = 4$. Οι σχετικά πρώτοι αριθμοί είναι οι $\{1, 2, 3, 4\}$.
    \item Ισχύει $\phi(9) = 6$. Οι σχετικά πρώτοι αριθμοί είναι οι $\{1, 2, 4, 5, 7, 8\}$.
    \item Ισχύει $\phi(26) = 12$. Οι σχετικά πρώτοι αριθμοί είναι οι $\{1, 3, 5, 7, 9, 11, 15, 17, 19, 21, 23, 25\}$.
\end{enumerate}

\section*{Ερώτημα 4}

Στο τέταρτο και τελευταίο ερώτημα, θα πρέπει να θυμηθούμε τον \emph{Ομοπαραλληλικό Κρυπταλγόριθμο} από την προηγούμενη άσκηση. Με $n \in \mathbb{Z}^+$, και κλειδί $(a, b) : \gcd(a, n) = 1$, το απλό κείμενο $m$ συνδέεται με το κρυπτοκείμενο $c$ με την σχέση

$$c \equiv am + b \pmod n$$

Ένα σχήμα που αυξάνει το μέγεθος του κλειδιού, άρα και την ανθεκτικότητα του κρυπταλγορίθμου ενάντια στην κρυπτανάλυση είναι η λεγόμενη \emph{διπλή κρυπτογράφηση}. Στην διπλή κρυπτογράφηση, χρησιμοποιούνται δύο κλειδιά και το απλό κείμενο κρυπτογραφείται πρώτα με το πρώτο κλειδί, με το κρυπτοκείμενο να κρυπτογραφείται \emph{ξανά} με το δεύτερο κλειδί, παράγοντας το τελικό κρυπτοκείμενο.

Ας προσπαθήσουμε να εφαρμόσουμε το σχήμα αυτό στον κρυπταλγόριθμό μας. Θα πούμε τα δυο κλειδιά $(a_1, b_1)$ και $(a_2, b_2)$, και το ενδιάμεσο κρυπτοκείμενο $c_1$. Ας κάνουμε μερικές πράξεις:

\begin{align*}
    c_1 &\equiv a_1m + b_1 \pmod n \\
    c &\equiv a_2c_1 + b_2 \equiv a_2(a_1m + b_1) + b_2 \equiv \\
    &\equiv a_2a_1m + (a_2b_1 + b_2) \pmod n
\end{align*}

Παρατηρούμε στην τελευταία γραμμή ότι ο κρυπταλγόριθμός μας εφαρμοσμένος εις διπλούν με τα κλειδιά που είπαμε, ισοδυναμεί με τον κρυπταλγόριθμο εφαρμοσμένο άπαξ και με κλειδί την δυάδα $(a_2a_1, a_2b_1 + b_2)$. Αυτό σημαίνει ότι αντίθετα με την κοινή λογική, μια διπλή κρυπτογράφηση με τον Ομοπαραλληλικό Κρυπταλγόριθμο δεν προσφέρει επιπλέον ασφάλεια. Επίσης, το κλειδί θα είναι πάντα έγκυρο αφού το $a_2a_1$ είναι σχετικά πρώτο με το $n$, μιας και αμφότερα τα $a_1$ και $a_2$ είναι σχετικά πρώτα με το $n$.

\subsection*{Ένα παράδειγμα}

Ας δούμε ένα παράδειγμα. Έχουμε τα κλειδιά $(3, 5)$ και $(11, 7)$ και $n = 26$. Σύμφωνα με την παραπάνω πρόταση, το ισοδύναμο κλειδί θα είναι

$$(11 \cdot 3, 11 \cdot 5 + 7) = (33, 62) \equiv (7, 10) \pmod{26}$$

Ας κρυπτογραφήσουμε και ένα γράμμα για να είμαστε σίγουροι. Έστω $m = \mathtt{K} \equiv 10$. Θα κρυπτογραφήσουμε το $m$ δυο φορές όπως γίνεται συνήθως η διπλή κρυπτογράφηση:

\begin{gather*}
    c_1 \equiv 3 \cdot 10 + 5 \equiv 35 \equiv 9 \pmod{26} \\
    c \equiv 11 \cdot 9 + 7 \equiv 106 \equiv 2 \pmod{26} \\
    c \equiv \mathtt{C}
\end{gather*}

Τώρα, θα κρυπτογραφήσουμε το $m$ μια φορά με το ισοδύναμο κλειδί που βρήκαμε προηγουμένως:

$$c \equiv 7 \cdot 10 + 10 \equiv 80 \equiv 2 \pmod{26}$$

Παρατηρούμε ότι και στις δύο περιπτώσεις, το γράμμα \texttt{K} κρυπτογραφείται στο γράμμα \texttt{C}.

\subsection*{Μια Επίθεση Εξαντλητικής Αναζήτησης}

Χάρη στην ισοδυναμία που δείξαμε προηγουμένως, αν θέλουμε να πραγματοποιήσουμε μια επίθεση Εξαντλητικής Αναζήτησης στον διπλό Ομοπαραλληλικό Κρυπταλγόριθμο, μπορούμε να τον αντιμετωπίσουμε ως έναν απλό Ομοπαραλληλικό Κρυπταλγόριθμο και να δοκιμάσουμε όλα τα κλειδιά αντί για όλες \emph{τις δυάδες κλειδιών}. Αυτό σημαίνει ότι ουσιαστικά ο κλειδοχώρος του Ομοπαραλληλικού Κρυπταλγορίθμου \emph{δεν} αυξάνεται με την διπλή κρυπτογράφηση (άρα επαγωγικά ούτε και με την πολλαπλή).

\begin{thebibliography}{1}
    \bibitem{inversemod}
    \texttt{\url{https://planetcalc.com/3311/}}
\end{thebibliography}

\end{document}