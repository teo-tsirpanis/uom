\documentclass{article}

\usepackage{fontspec}
\usepackage[greek,english]{babel}

\setmainfont{Ubuntu}
\setmonofont{Ubuntu Mono}

\title{Βάσεις Δεδομένων I -- Άσκηση 2}
\author{Τσιρπάνης Θεόδωρος\\ \texttt{dai19090}}
\date{Νοέμβριος 2019}

\begin{document}

\selectlanguage{greek}

\maketitle

\section*{Συμβολογραφία}

\begin{itemize}
\item Για τους πίνακες θα χρησιμοποιείται ο συμβολισμός \texttt{όνομα\_πίνακα}(\texttt{πεδία}).

\item Τα προτεύοντα κλειδιά ενός πίνακα θα \underline{υπογραμμίζονται}.

\item Τα ξένα κλειδιά που αναφέρονται στο πεδία \texttt{x} του πίνακα \texttt{A} θα συμβολίζονται ως \texttt{A.x}.

\item Τιμές που είναι οπωσδήποτε προαιρετικές (μπορούν να πάρουν την τιμή \texttt{NULL}) συμβολίζονται με \textit{πλάγια γράμματα}.
Να σημειωθεί ότι τα δοθέντα διαγράμματα ER δεν συμβολίζουν αν η ιδιότητα μιας οντότητας που δεν είναι κλειδί μπορεί να πάρει την τιμή \texttt{NULL}· θα υποθέσουμε ότι δεν μπορεί.

\item Ένα πεδίο μπορεί να μετονομαστεί από \texttt{x} σε \texttt{y} γράφοντας \texttt{x:y}.
\end{itemize}


\section{Ένα απλό παράδιγμα}

Ε1(\underline{a1, a2, E2.b1, E2.b2}, a3) \\
E2(\underline{b1, b2}) \\
E4(\underline{d1, E5.f1}, \textit{c1, E2.b1, E2.b2}) \\
E5(\underline{f1}, f2) \\
R3(\underline{E4.d1, E5.f1, E4.d1:d2, E5.f1:f2})

\section{Η εταιρεία}

Πόστο(\underline{κωδικός}, όνομα) \\
Υποκατάστημα(\underline{πόλη, διεύθυνση}) \\
Υπάλληλος(\underline{ΑΦΜ}, \textit{Υποκατάστημα.πόλη, Υποκατάστημα.διεύθυνση}, όνομα, επίθετο, Υπάλληλος.ΑΦΜ:διαθέτει\_ΑΦΜ) \\
Έχει(\underline{Πόστο.κωδικός, Υποκατάστημα.πόλη, Υποκατάστημα.διεύθυνση}) \\
Εργάζεται\_Σε(\underline{Πόστο.κωδικός, Υποκατάστημα.πόλη, Υποκατάστημα.διεύθυνση, Υπάλληλος.ΑΦΜ})

\section{Το ποδόσφαιρο}

Προπονητής(\underline{ΑΦΜ}, ονοματεπώνυμο, εθνικότητα) \\
Ομάδα(\underline{κωδικός, Προπονητής.ΑΦΜ}, όνομα, έδρα, περιοχή) \\
Παίκτης(\underline{κωδικός}, ονοματεπώνυμο, έτος\_γέννησης, \textit{Ομάδα.κωδικός, αριθμός\_φανέλας}) \\
Αγώνας(\underline{Ομάδα.κωδικός:γηπεδούχος, Ομάδα.κωδικός:φιλοξενούμενος, ημερομηνία}, αποτέλεσμα) \\
Συμμετέχει(\underline{Παίκτης.κωδικός, Αγώνας.γηπεδούχος, Αγώνας.φιλοξενούμενος, Αγώνας.ημερομηνία}, χρόνος, γκολ) \\

Σε επίπεδο εφαρμογής θα πρέπει να διασφαλιστεί ότι μια ομάδα έχει τουλάχιστον έναν παίκτη.

\end{document}