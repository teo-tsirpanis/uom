\documentclass{article}

\usepackage{amsmath}
\usepackage{fontspec}
\usepackage[greek,english]{babel}

\setmainfont{Ubuntu}
\setmonofont{Ubuntu Mono}

\title{Βάσεις Δεδομένων I -- Άσκηση 5}
\author{Τσιρπάνης Θεόδωρος\\ \texttt{dai19090}}
\date{Ιανουάριος 2020}

\begin{document}

\selectlanguage{greek}

\maketitle

\section*{Άσκηση 1}

Έχουμε τον πίνακα R(A, B, C, D, E, F). Το μοναδικό υποψήφιο κλειδί είναι το σύνολο $\{B, F\}$. Θα τον κανονικοποιήσουμε πρώτα σε 2NF.

\subsection*{2NF}

Ο R δεν είναι σε 2NF. Με τις σχέσεις $B \rightarrow C$ και $C \rightarrow A$, θα τον διασπάσουμε, δημιουργώντας τους πίνακες R1(\underline{B}, D, E, \underline{F}) και R2(A, \underline{B}, C).

Ο R2 είναι σε 2NF, αλλά ο R1 δεν είναι. Θα χρησιμοποιήσουμε την σχέση $F \rightarrow D$, για να τον διασπάσουμε στους πίνακες R11(\underline{B}, E, \underline{F}) και R12(D, \underline{F}). Όλοι οι πίνακες είναι τώρα σε 2NF.

\subsection*{3NF}

Επιπλέον, όλοι οι πίνακες εκτός από τον R2 είναι σε 3NF. Η μετατροπή του είναι εκπληκτικά απλή. Θα τον διασπάσουμε με την σχέση $C \rightarrow A$, στους πίνακες R21(\underline{B}, C) και R22(A, \underline{C}). Οπότε το τελικό μας σχήμα αποτελείται από τους πίνακες R11, R12, R21 και R22.

\subsection*{BCNF}

Το σχήμα μας είναι ήδη σε BCNF. Αν παρατηρήσουμε μάλιστα, υπάρχουν τέσσερεις πίνακες, και τέσσερεις εξαρτήσεις, με το κλειδί του καθενός να αντιστοιχεί με το αριστερό μέλος μιας σχέσης.

\section*{Άσκηση 2}

Έχουμε τον πίνακα R(A, B, C, D, E). Τα υποψήφια κλειδιά είναι τα $\{\{B\}, \{C\}, \{A, E\}, \{D, E\}\}$. Το σχήμα είναι σε 2NF και 3NF, μιας και όλα τα πεδία του ανήκουν σε κάποιο υποψήφιο κλειδί.

\subsection*{BCNF}

Ο πίνακας δεν είναι σε BCNF και δεν μπορεί να γίνει επειδή σχηματίζεται ένας κύκλος από τις εξαρτήσεις του. Η επίλυση του προβλήματος αυτού είναι εκπληκτικά απλή. Θα αφαιρέσουμε μια εξάρτηση (εδώ την $C \rightarrow B$), καταλήγοντας στο μόνο υποψήφιο κλειδί να είναι το $\{B\}$. Το καινούριο μας σχήμα θα κανονικοποιηθεί στους πίνακες R1(A, \underline{B}, E), R2(\underline{A}, D), και R3(C, \underline{D}, \underline{}{E}).

\section*{Άσκηση 3}

Έχουμε τον πίνακα R(αριθμός\_παραγγελίας, κωδικός\_πελάτη, ημερομηνία\_πώλησης, κωδικός\_πωλητή, ονοματεπώνυμο\_πωλητή, τηλέφωνο\_πωλητή, ονοματεπώνυμο\_πελάτη, διεύθυνση\_πελάτη, κωδικός\_προϊόντος, περιγραφή\_προϊόντος, ποσότητα\_προϊόντος, τιμή\_μονάδας\_προϊόντος). Το υποψήφιο κλειδί του είναι το $\{\text{αριθμός\_παραγγελίας}, \text{κωδικός\_προϊόντος}\}$. Το σχήμα δεν είναι σε 2NF, και δεν θα προσπαθήσουμε να το μετατρέψουμε σε αυτήν την μορφή. Αντ' αυτού, θα το μετατρέψουμε κατευθείαν σε BCNF. Το νέο μας σχήμα αποτελείται από τους εξής πίνακες:

\begin{itemize}
    \item Παραγγελίες(αριθμός\_παραγγελίας, κωδικός\_προϊόντος)
    \item Στοιχεία\_παραγγελίας(\underline{αριθμός\_παραγγελίας} κωδικός\_πελάτη, ημερομηνία\_πώλησης, κωδικός\_πωλητή)
    \item Στοιχεία\_πελάτη(\underline{κωδικός\_πελάτη}, ονοματεπώνυμο\_πελάτη, διεύθυνση\_πελάτη)
    \item Στοιχεία\_πωλητή(\underline{κωδικός\_πωλητή}, ονοματεπώνυμο\_πωλητή, τηλέφωνο\_πωλητή)
    \item Στοιχεία\_προϊόντος(\underline{κωδικός\_πωλητή}, ονοματεπώνυμο\_πωλητή, τηλέφωνο\_πωλητή)
    \item Ποσότητες\_παραγγελίας(\underline{αριθμός\_παραγγελίας}, \underline{κωδικός\_προϊόντος}, ποσότητα\_προϊόντος)
\end{itemize}

\end{document}
