\documentclass{beamer}

\usepackage{hyperref}
\usepackage{cancel}
\usepackage{fontspec}
\usepackage{longtable}
\usepackage[greek,english]{babel}

\setsansfont{Ubuntu}
\setmonofont{Ubuntu Mono}

\title{Κρυπτογραφία -- Γεννήτριες τυχαίων αριθμών}
\author[Τσιρπάνης Θεόδωρος]{Τσιρπάνης Θεόδωρος\\ \texttt{dai19090@uom.edu.gr}}
\institute[ΠΑΜΑΚ]{Πανεπιστήμιο Μακεδονίας -- Τμήμα Εφαρμοσμένης Πληροφορικής}
\date{Δεκέμβριος 2019}

\begin{document}

\selectlanguage{greek}

\frame{\titlepage}

\begin{frame}{Τι είναι ένας τυχαίος αριθμός;}
    \begin{itemize}
        \item<2-> Είναι το 7 τυχαίος αριθμός;
        \item<3-> Το 45;
        \item<4-> Το 521;
        \item<5-> Το 3141592653589;
        \item<6-> Από μόνος του ένας αριθμός δεν είναι τυχαίος!
    \end{itemize}
\end{frame}

\begin{frame}{Ακολουθίες τυχαίων αριθμών}
	\begin{itemize}
    \item<1-> Επομένως μιλάμε για \alert{ακολουθίες} τυχαίων αριθμών.
    \item<2-> Επειδή θέλουμε πολλές τέτοιες ακολουθίες, μιλάμε για \alert{συναρτήσεις} που \alert{παράγουν} ακολουθίες τυχαίων αριθμών.
	\item<3-> Ορίζουμε $a(s)_n$ τον $n$-οστό όρο μιας ακολουθίας τυχαίων αριθμών, με τιμή αρχικοποίησης το $s$. Γνωρίζοντας τον αλγόριθμο (γεννήτρια) και το $s$, μπορούμε να παράγουμε ολόκληρη την ακολουθία.
	\item<4-> Το $s$ αποκαλείται συχνά και \alert{σπόρος (seed)} της ακολουθίας. Ο σπόρος συνήθως είναι μια τιμή \textit{μοναδική}, όχι τυχαία απαραίτητα.
    \end{itemize}
\end{frame}

\begin{frame}{Είδη ακολουθιών τυχαίων αριθμών}
	Ο Bruce Schneier στο βιβλίο του \textit{Applied Cryptography}, χωρίζει τις ακολουθίες τυχαίων αριθμών σε τρεις κατηγορίες:
	
	\begin{itemize}
	\item<2-> \textit{Ψευδοτυχαίες ακολουθίες}
	\item<3-> \textit{Κρυπτογραφικά ασφαλείς ψευδοτυχαίες ακολουθίες}
	\item<4-> \textit{Πραγματικά τυχαίες ακολουθίες}
	\end{itemize}
	
	\onslide<5->{Θα μιλήσουμε για την κάθε μία στη συνέχεια.}
\end{frame}

\begin{frame}{Ψευδοτυχαίες ακολουθίες}
	\begin{itemize}
		\item<2-> Μια ακολουθία ονομάζεται \alert{ψευδοτυχαία} όταν \textit{φαίνεται} πως είναι τυχαία.
		\item<3-> Το ``φαίνεται'' το καταλαβαίνουμε πιο αυστηρά με \alert{στατιστικά τεστ τυχαιότητας}:
		\begin{itemize}
			\item<4-> Ομοιόμορφη κατανομή στα bits, bytes, λέξεις των 32 bit.
			\item<5-> Μεγάλη περίοδος.
			\item<6-> Υπάρχουν πακέτα λογισμικού που κάνουν τέτοια τεστ, όπως το \href{https://en.wikipedia.org/wiki/Diehard_tests}{Diehard} και το \href{https://en.wikipedia.org/wiki/TestU01}{TestU01}.
		\end{itemize}
		\item<7-> Η στατιστική τυχαιότητα είναι χρήσιμη σε προσομοιώσεις, κληρώσεις και παιχνίδια μεταξύ άλλων.
	\end{itemize}
\end{frame}

\begin{frame}{Η γεννήτρια LCG}{Linear Congruential Generator}
	\begin{itemize}
		\item <2-> $X_{n + 1} = aX_n + c \pmod m$. To $X_0$ είναι ο σπόρος.
		\item <3-> Θα δούμε περιπτώσεις όπου το $m$ είναι δύναμη του 2.
		\item <4-> Χρησιμοποιείται στις C, C++, Java και Visual Basic 6.
		\item <5-> \textcolor{green}{Πολύ μικρές απαιτήσεις χρόνου και μνήμης}
		\item <6-> \textcolor{green}{Δυνατότητα τυχαίας προσπέλασης προς τις δύο κατευθύνσεις}
		\item <7-> \textcolor{green}{Επιλέγοντας το $c$ (εφόσον είναι σχετικά πρώτος με το $m$), μπορούμε να έχουμε πολλές τυχαίες ακολουθίες με τον ίδιο σπόρο}
		\item <8-> \textcolor{red}{Στατιστικά προβλήματα}
	\end{itemize}
\end{frame}

\begin{frame}{Συχνές παράμετροι για τον LCG}{Πηγή: Βικιπαίδεια}
	\begin{table}
	\begin{tabular}[c]{l | c | c | c | c }
		& $m$ & $a$ & $c$ & Bits \\
		\hline
		Standard C & $2^{31}$ & $1103515245$ & $12345$ & 30..16 \\
		Turbo Pascal & $2^{32}$ & $134775813$ & $1$ & \\
		Java & $2^{48}$ & $25214903917$ & $11$ & 47..16 \\
		Visual Basic 6 & $2^{24}$ & $1140671485$ & $12820163$ & \\
	\end{tabular}
	\end{table}
\end{frame}

\begin{frame}{Άλλες γεννήτριες ψευδοτυχαίων αριθμών}
	\begin{itemize}
		\item<2-> PCG -- Μια σχετικά καινούργια γεννήτρια, βασίζεται στoν LCG, εξαιρετικές στατιστικές ιδιότητες.
		\item<3-> Mersenne Twister -- Χρησιμοποιείται στις Python, Free Pascal, Ruby, PHP μεταξύ άλλων. Παρά την τεράστια περίοδο ($2^{19937} - 1$), έχει μεγάλες απαιτήσεις χώρου, και αποτυγχάνει ορισμένα στατιστικά τεστ.
		\item<4-> LFSR -- Χρησιμοποιούνταν ιστορικά για στρατιωτικούς σκοπούς. Εύκολα υλοποιήσιμή σε Hardware, αλλά ξεπερασμένη.
	\end{itemize}
\end{frame}

\begin{frame}{Κρυπτογραφικά ασφαλείς ψευδοτυχαίες ακολουθίες}{Cryptographically Secure Pseudorandom Number Generators (CSPRNGs)}
	\begin{itemize}
		\item<2-> Μια ψευδοτυχαία ακολουθία ονομάζεται \alert{κρυπτογραφικά ασφαλής}, όταν είναι \alert{απρόβλεπτη}.
		\item<3-> Αυτό σημαίνει ότι δεν μπορούμε να προβλέψουμε τον επόμενο αριθμό της, αν γνωρίζουμε όλους τους προηγούμενους.
		\item<4-> Ορισμένες φορές βασίζονται σε κρυπτογραφικές συναρτήσεις (block ciphers).
		\item<5-> Ο σπόρος τους είναι μεγαλύτερου μεγέθους από τις ψευδοτυχαίες ακολουθίες.
	\end{itemize}
\end{frame}

\begin{frame}{Εφαρμογές κρυπτογραφικά ασφαλών ακολουθιών}
    \begin{itemize}
		\item<2-> Δημιουργία κρυπτογραφικών κλειδιών.
		\item<3-> Συμμετρικοί Κρυπταλγόριθμοι Ροής (Stream Ciphers)
		\begin{itemize}
			\item<4-> Δεν χωρίζουν το καθαρό κείμενο σε τμήματα.
			\item<5-> Σχεδόν πάντα είναι CSPRNGs, όπου το κρυπτοκείμενο είναι η έξοδός τους XOR το καθαρό κείμενο.
			\item<6-> \textcolor{green}{Αυτό σημαίνει ότι η αποκρυπτογράφηση είναι η ίδια πράξη με την κρυπτογράφηση.}
			\item<7-> \textcolor{red}{Όμως, η χρήση τους πρέπει να γίνει με πολύ μεγάλη προσοχή (όπως να μην χρησιμοποιηθεί το ίδιο κλειδί πολλές φορές, όπως τα one-time pads).}
			\item<8-> Παραδείγματα: \cancel{RC4} (εξαιρετικά κρυπτογραφικά αδύναμος), Salsa20, ChaCha 20.
		\end{itemize}
	\end{itemize}
\end{frame}

\begin{frame}{Πραγματικά τυχαίες ακολουθίες}
    \begin{itemize}
        \item<2-> Μια κρυπτογραφικά ασφαλής ακολουθία ονομάζεται \alert{πραγματικά τυχαία}, όταν \alert{δεν μπορεί να αναπαραχθεί με αξιοπιστία}.
        \item<3-> Οι υπολογιστές είναι αιτιοκρατικές μηχανές, και δεν μπορούν να δημιουργήσουν \textit{πραγματικά} πραγματικά τυχαίους αριθμούς.
        \item<4-> Αντί γι' αυτό, αντλούν μεγάλες πηγές τυχαιότητας από περιβαλλοντικό θόρυβο (όπως την θερμοκρασία της CPU, και περιφερειακές συσκευές)· πρακτικά, δεν μπορούν να αναπαραχθούν οι ίδιες ακριβώς συνθήκες.
	\end{itemize}
\end{frame}

\begin{frame}{Γεννήτριες πραγματικά τυχαίων αριθμών}
	\begin{itemize}
        \item<2-> Στα συστήματα UNIX υπάρχει το αρχείο \texttt{/dev/random}.
        \item<3-> H Java έχει την κλάση \texttt{SecureRandom}, η C\# την \texttt{RandomNumberGenerator}, ενώ η Python την συνάρτηση \texttt{os.urandom}.
        \item<4-> Πρόσφατες CPU της αρχιτεκτονικής x86 έχουν την οδηγία \texttt{RDRAND} που φορτώνει έναν πραγματικά τυχαίο αριθμό σε έναν καταχωρητή.
        \item<5-> Εξειδικευμένο hardware και ιστοσελίδες όπως το \href{https://random.org}{random.org} δημιουργούν πραγματικά τυχαίους αριθμούς από κβαντικά φαινόμενα, άρα και πραγματικά απρόβλεπτους.
	\end{itemize}
\end{frame}

\begin{frame}
	\huge{Ερωτήσεις;}
\end{frame}

\end{document}